\section{Einleitung}
\subsection{Ausgangslage}
Evolutionäre Algorithmen sind Optimierungsverfahren die sich, inspiriert durch in der Natur ablaufende Prozesse, den grundlegenden evolutionären Mechanismen von Mutation, Selektion und Rekombination bedienen. In dieser Arbeit werden verschiedene Selektionsstrategien an einem einfachen Beispiel miteinander verglichen.

\subsection{Ziel der Arbeit}
Ziel der Arbeit ist es, an einem einfachen Beispiel, verschiedene Ansätze zum Entwurf von relativen Fitnessfunktionen gegeneinander abzuwägen. Dazu werden evolutionäre Algorithmen mit unterschiedlichen Strategien implementiert. Die aus den unterschiedlichen Ansätzen resultierenden Lösungen werden dann hinsichtlich ihrer Qualität und der für ihre Suche benötigten Ressourcen (Rechenzeit) verglichen.

\subsection{Aufgabenstellung}
In dieser Arbeit sollen evolutionäre Algorithmen implementiert welchen, welche zu einem gegebenen Regulären Ausdruck einen entsprechenden endlichen Automaten finden. Für dieses konkrete Beispiel sollen evolutionäre Algorithmen mit verschiedenen Strategien implementiert werden, umso das Konvergenzverhalten und die Rechenzeit der verschiedenen Implementationen (und somit der Strategien) untersuchen und vergleichen zu können.
\begin{itemize}
	\item Einarbeitung in das Thema
	\item Repräsentation und Mutation von endlichen Automaten definieren 
	\item Evolutionärer Algorithmus mit fester, globaler Problemmenge implementieren 
	\item Evolutionärer Algorithmus mit mutierender, globaler Problemmenge implementieren 
	\item Evolutionärer Algorithmus, bei welchem jeder Lösungskandidat seine eigene, mutierende Problemmenge mitführt implementieren
	\item Gegenüberstellung der Resultate
\end{itemize}

\subsection{Erwartete Resultate}
\begin{itemize}
	\item Überblick über Evolutionäre Algorithmen und Endliche Automaten im technischen Bericht
	\item Implementierung einer Repräsentation von endlichen Automaten, Implementierung von Mutationen zu endlichen Automaten und Dokumentation des Implementierten im technischen Bericht 
	\item Implementierung eines evolutionären Algorithmus mit fester, globaler Problemmenge mit einer entsprechenden Dokumentation im technischen Bericht 
	\item Implementierung eines evolutionären Algorithmus mit mutierender, globaler Problemmenge mit einer entsprechenden Dokumentation im technischen Bericht 
	\item Implementierung eines evolutionären Algorithmus, bei welchem jeder Lösungskandidat seine eigene, mutierende Problemmenge mitführt mit einer entsprechenden Dokumentation im technischen Bericht 
	\item Dokumentation der Gegenüberstellung der Resultate
\end{itemize}
