\clearpage
\section{Projektfazit}
Trotz anfänglicher Bedenken aufgrund der Tatsache, dass sich die Funktionsweise eines endliche Automaten selbst bei kleinsten Veränderungen komplett ändern kann, ist es mithilfe von evolutionären Algorithmen gelungen solche Automaten gezielt und mit endlichem Rechenaufwand zu generieren. Auch konnten Unterschiede im Konvergenzverhalten der verschiedenen Algorithmen, sowohl bei unterschiedlichen Problemstellungen, als auch bei den verschiedenen Ansätzen zum Umgang mit Problemmengen festgestellt, analysiert und ergründet werden. Es wurde festgestellt, dass der Rechenaufwand beim finden von endlichen Automaten zu gegebenen regulären Ausdrücken mithilfe eines evolutionären Algorithmus mit einer einer globalen, mutierenden Problemmenge geringer ist als bei den beiden anderen untersuchten Ansätzen. Ob und falls ja, wieweit dieses Ergebnis generalisiert werden kann war nicht Ziel dieser Arbeit und müsste in weiteren Studien untersucht werden. Ich bedanke mich herzlich bei Dandolo Flumini. Er hat mich mit guten Ideen beliefert und so zum Erfolg dieses Projektes beigetragen.