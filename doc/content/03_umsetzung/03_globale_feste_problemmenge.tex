\subsubsection{Feste globale Problemmenge}
Als \flqq Grundversion\frqq und Grundlage für die anderen Implementationen wurde zuerst ein evolutionärer Algorithmus mit einer festen, globalen Problemmenge implementiert.

Der Algorithmus an sich funktioniert hierbei grob wie folgt: 
\begin{enumerate}
\item Initialisierung (Sprache, Problemmenge, Population, Referenzautomat)
\item Berechnen der Fitness aller Individuen
\item Sortieren der Individuen nach Fitness
\item Hat ein Individuum alle Probleme korrekt gelöst? Falls ja, wird es mit dem Referenzautomaten verglichen. Wenn es einer korrekten Lösung entspricht wird der Algorithmus angehalten und die Anzahl benötigter Zyklen wird zurückgegeben.
\item Von den besten 50\% der Individuen wird eine \Gls{deepcopy} erstellt
\item Die Kopien werden zufällig mutiert
\item Die besten 50\% und deren mutierte Kopien bilden die neue Population
\item Die Variable zum zählen der Zyklen wird um 1 erhöht.
\item Wenn das Zykluslimit noch nicht überschritten ist, weiter mit 2. ansonsten war der Algorithmus nicht erfolgreich und bricht ab.
\end{enumerate}